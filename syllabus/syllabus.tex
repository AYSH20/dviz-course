\documentclass[11pt,article,oneside]{memoir}

\usepackage{org-preamble-pdflatex} 

\setlength{\parskip}{10pt}
\setlength{\parindent}{0pt}

% Definitions
\def\myauthor{Author}
\def\mytitle{Title}
\def\mycopyright{\myauthor}
\def\mykeywords{}
\def\mybibliostyle{plain}
\def\mybibliocommand{}
\def\mysubtitle{}
\def\myaffiliation{Indiana University}
\def\myaddress{Info East Rm 316} 
\def\myemail{yyahn@indiana.edu}
\def\myweb{http://yongyeol.com}
\def\myphone{856-2920}
\def\myversion{}
\def\myrevision{}

\def\myaffiliation{\ \\Indiana University}
\def\myauthor{Yong-Yeol (YY) Ahn}
\def\mykeywords{Visualization, Data, Undergraduate, Informatics}
\def\mysubtitle{Syllabus}
\def\mytitle{{\normalsize \textsc{Info} 422 (H400/I590) \newline} \HUGE Data Visualization}

\begin{document}

%%\chapterstyle{article-3}
%\pagestyle{kjh}

\def\ind{\hangindent=1 true cm\hangafter=1 \noindent}
\def\labelitemi{$\cdot$}

\chapterstyle{article-4}  % alternative styles are defined in latex-custom-kjh/needs-memoir/

\title{\LARGE \mytitle}     
\author{\Large\myauthor \newline \footnotesize\texttt{\noindent\myemail}}
\date{Fall 2015. Info West 107 (M) / 109 (W)\newline MW 4:00pm--5:15pm. \newline Office hours: T 9am-10:45am}

%\published{\sffamily I590/H400/I400 / Fall 2014 / Mon \& Wed 4:00--5:15pm / Info West 107 (M) \& 109 (W)}
\maketitle

\vspace{-20pt}
{\bfseries Assistant Instructor} \\ Qing Ke (\texttt{qke@indiana.edu}) \\ Office Hours: WF 10:10am-11am

\section{Course Description}

From dashboards in a car to cutting-edge scientific papers, we extensively use
visual representation of data. As our world becomes increasingly connected and
digitized, and as decisions are increasingly driven by data, data visualization
is becoming a critical skill for every knowledge worker. This course is an
introduction to basic data analysis and visualization. We will learn
fundamentals of data visualization in terms of exploratory visualization
(Python stack) and explanatory visualization (Javascript and D3.js).  


\section{Course Objectives}

By the end of the course, you will be able to understand various types of data,
explore datasets using basic exploratory visualization techniques, and create
simple explanatory web-based visualizations. You will also be able to evaluate
the effectiveness of data visualizations based on the principles of human
perception, design, and types of data. 
 

%\begin{itemize}
    %\item Exploratory data visualization and analysis

%\begin{itemize}
%\item Python data analysis stack
%\item Data types 
%\item Loading and manipulating data 
%\item Fundamental data visualizations
%\end{itemize}

%\end{itemize}

\section{Prerequisites}
\label{sec:Prerequisites}

This course is open to advanced undergraduate students (I422/H400) as well as
graduate students (I590). Because this course uses programming languages
(Python and Javascript) as the main tool, it is required to have good
understanding and working knowledge of programming.  The basic programming
courses (I210 \& I211, or equivalent) are required prerequisites. In addition,
I308: ``Information Representation'' is a recommended prerequisite. Basic
understanding of statistics and web (HTML, CSS, Javascript) will be very
helpful. 

%For self-assessment, visit the following link:
%\begin{itemize}
    %\item  \href{http://bit.ly/dviz\_selfassess}{http://bit.ly/dviz\_selfassess}
%\end{itemize}

%If you feel comportable with those questions, you will not have difficulties.
Contact the instructor if you are uncertain about your background. 

\section{Requirements}
\label{sec:requirements}

You should attend and participate in class. There will be reading materials
that you are expected to read \emph{prior} to the class. At the beginning of
each class, there will be an in-class quiz based on assigned readings and
materials from previous classes.  

There will be several class assignments. One missed assignment will not count
against you, but no late assignments will be accepted. 

The mid-term and final assessment will be project-based. For the mid-term
project, you will perform exploratory data analyses on datasets of your choice
(which will be set through discussion with the instructors). The final
assessment will be about creating web-based explanatory visualization that
communicate insights into datasets (ideally from the datasets that you analyzed
at the mid-term). 

\section{Books and key materials}

There is no required textbook, but the following books and websites are recommended.

\subsection{Python and data analysis}

\begin{enumerate}

\item \href{http://www.diveintopython3.net/index.html}{Dive Into Python} by Mark Pilgrim (available online): a good Python book. 

\item \href{http://www.learnpython.org}{Learnpython.org}: A web-based interactive tutorial. 

\item \href{http://work.thaslwanter.at/Stats/html/}{An introduction to statistics} (with Python) by Thomas Haslwanter (available online): this book uses Python to explain basic statistics. It also contains a succinct tutorial for Python and data visualization using Python. 

\item \href{http://ipython.rossant.net}{Learning IPython for Interactive Computing and Data Visualization} by  Cyrille Rossant: Introduction to IPython as well as lots of advanced analysis 


\end{enumerate}

\subsection{Visualization and Design}

\begin{enumerate}

\item \href{http://www.amazon.com/gp/product/0961392142}{The Visual Display of Quantitative Information (2nd ed.)} by E.R. Tufte: one of the foundational book on visualization. It contains a rich set of historical visualization, thoughtful discussion on visualization principles. 

\item \href{http://www.amazon.com/Atlas-Knowledge-Anyone-Can-Map/dp/0262028816}{Atlas of Knowledge: Anyone Can Map} by K. Börner: this book systematically analyzes vocabularies of visualization with a lot of great examples. 

\item \href{http://www.amazon.com/Visualization-Analysis-Design-AK-Peters/dp/1466508914}{Visualization Analysis and Design} by T. Munzner: a nice textbook that covers important topics of visualization. 

\item \href{http://www.amazon.com/Visual-Thinking-Kaufmann-Interactive-Technologies/dp/0123708966}{Visual Thinking for Design} by C. Ware: one of the best books on the role of perception in visualization. 

\end{enumerate}

\subsection{D3.js}

\begin{enumerate}

\item \href{http://www.amazon.com/Interactive-Data-Visualization-Scott-Murray/dp/1449339735}{Interactive Data Visualization for the Web} by Scott Murray

\item \href{https://github.com/mbostock/d3/wiki}{D3 documentation}

\end{enumerate}

\section{Policies}

\begin{enumerate}

\item \emph{Disabilities.} Every attempt will be made to accommodate qualified
students with disabilities (e.g. mental health, learning, chronic health,
physical, hearing, vision, neurological, etc.). You must have established your
eligibility for support services through Disability Services for Students. Note
that services are confidential, may take time to put into place, and are not
retroactive.  Captions and alternate media for print materials may take three
or more weeks to get produced. Please contact Disability Services for Students
at \url{http://disabilityservices.indiana.edu} or 812-855-7578 as soon as
possible if accommodations are needed. The office is located on the third
floor, west tower, of the Wells Library (Room W302). Walk-ins are welcome 8 AM
to 5 PM, Monday through Friday. You can also locate a variety of campus
resources for students and visitors who need assistance at
\url{http://www.iu.edu/~ada/index.shtml}. 

\item \emph{No electronics---laptops, tablets, and smartphones---may be used in
class}, unless the usage is specifically requested by the instructors.
\href{http://www.scientificamerican.com/article/a-learning-secret-don-t-take-notes-with-a-laptop/}{It
has been shown} that using laptops in class is not a good idea, \emph{even if}
you are using it to take notes.  If you must have electronics due to a
disability or other special reasons, please contact the instructor. 

\item \emph{Be honest.} Your assignments and papers should be your own work.
First, if you find useful resources for your assignments, share them and cite
them. If your friends helped you, acknolwedge them. Second, feel free to
discuss both online and offline, but you should not show your code (papers) nor
see other's. Any cases of academic misconduct (cheating, fabrication,
plagiarism, etc) will be immediately reported to the School and the Dean of
Students, following the standard procedure. 

\item \emph{You have the responsibility of backing up all your data and code}.
Always use at least Box, Dropbox, or Google Drive. Ideally, learn version
control systems and use \url{https://github.iu.edu} or
\url{https://github.com}. Loss of data, code, or papers is not an acceptable
excuse for delayed or missing submission. 

\item \emph{Inform your excused absences prior to class}. Please contact the
instructor until the previous day for an excused absence.  

\item If you have any issues, don't hesistate to contact me or
\href{http://healthcenter.indiana.edu/counseling/index.shtml}{IU's Counseling
and Psychological Services}.


\end{enumerate}

\section{Grading (Tentative)}
\label{sec:grading_tentative_}

\begin{itemize}

\item Attendance, Quiz, and Participation (in-class and online): 30\%

\item Assignments (class and lab): 30\%

\item Mid-term and Final project: 40\%


\end{itemize}

\section{Course Schedule}

\subsection{Week 1}

\vspace{-5pt}
\begin{itemize}
\itemsep=-5pt
\item 8/24: Introduction: why visualization?
\item 8/26: Introduction to Visualization tools and Python stack 
\end{itemize}

\subsection{Week 2}
\vspace{-5pt}
\begin{itemize}
\itemsep=-5pt
\item 8/31: Historical perspectives
\item 9/2: IPython environment and Python review
\end{itemize}

\subsection{Week 3}
\vspace{-5pt}
\begin{itemize}
\itemsep=-5pt
\item 9/7: \emph{Labor day}
\item 9/9: Python data visualization tools \& historical visualizations
\end{itemize}

\subsection{Week 4 (9/14, 9/16) Perception and Design}
\subsection{Week 5 (9/21, 9/23) Data Types and Fundamental Visualizations}
\subsection{Week 6 (9/28, 9/30) Fundamental Visualizations II }
\subsection{Week 7 (10/5, 10/7) Geospatial Data and Maps I}
\subsection{Week 8 (10/12, 10/14) Geospatial Data and Maps II}

\subsection{Week 9}
\vspace{-5pt}
\begin{itemize}
\itemsep=-5pt
\item 10/19: Midterm project presentation
\item 10/21: Web review (HTML, CSS, Javascript)
\end{itemize}

\subsection{Week 10 (10/26, 10/28) SVG and Javascript }
\subsection{Week 11 (11/2, 11/4) D3.js}
\subsection{Week 12 (11/9, 11/11) Graphs and Trees}
\subsection{Week 13 (11/16, 11/18) Texts and Maps}
\subsection{Week 14 (11/23, 11/25) Thanksgiving}
\subsection{Week 15 (11/30, 12/2) Interactions}
\subsection{Week 16 (12/7, 12/9) Final project presentation}

\end{document}
